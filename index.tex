% Options for packages loaded elsewhere
\PassOptionsToPackage{unicode}{hyperref}
\PassOptionsToPackage{hyphens}{url}
\PassOptionsToPackage{dvipsnames,svgnames,x11names}{xcolor}
%
\documentclass[
  letterpaper,
  DIV=11,
  numbers=noendperiod]{scrreprt}

\usepackage{amsmath,amssymb}
\usepackage{iftex}
\ifPDFTeX
  \usepackage[T1]{fontenc}
  \usepackage[utf8]{inputenc}
  \usepackage{textcomp} % provide euro and other symbols
\else % if luatex or xetex
  \usepackage{unicode-math}
  \defaultfontfeatures{Scale=MatchLowercase}
  \defaultfontfeatures[\rmfamily]{Ligatures=TeX,Scale=1}
\fi
\usepackage{lmodern}
\ifPDFTeX\else  
    % xetex/luatex font selection
\fi
% Use upquote if available, for straight quotes in verbatim environments
\IfFileExists{upquote.sty}{\usepackage{upquote}}{}
\IfFileExists{microtype.sty}{% use microtype if available
  \usepackage[]{microtype}
  \UseMicrotypeSet[protrusion]{basicmath} % disable protrusion for tt fonts
}{}
\makeatletter
\@ifundefined{KOMAClassName}{% if non-KOMA class
  \IfFileExists{parskip.sty}{%
    \usepackage{parskip}
  }{% else
    \setlength{\parindent}{0pt}
    \setlength{\parskip}{6pt plus 2pt minus 1pt}}
}{% if KOMA class
  \KOMAoptions{parskip=half}}
\makeatother
\usepackage{xcolor}
\setlength{\emergencystretch}{3em} % prevent overfull lines
\setcounter{secnumdepth}{5}
% Make \paragraph and \subparagraph free-standing
\makeatletter
\ifx\paragraph\undefined\else
  \let\oldparagraph\paragraph
  \renewcommand{\paragraph}{
    \@ifstar
      \xxxParagraphStar
      \xxxParagraphNoStar
  }
  \newcommand{\xxxParagraphStar}[1]{\oldparagraph*{#1}\mbox{}}
  \newcommand{\xxxParagraphNoStar}[1]{\oldparagraph{#1}\mbox{}}
\fi
\ifx\subparagraph\undefined\else
  \let\oldsubparagraph\subparagraph
  \renewcommand{\subparagraph}{
    \@ifstar
      \xxxSubParagraphStar
      \xxxSubParagraphNoStar
  }
  \newcommand{\xxxSubParagraphStar}[1]{\oldsubparagraph*{#1}\mbox{}}
  \newcommand{\xxxSubParagraphNoStar}[1]{\oldsubparagraph{#1}\mbox{}}
\fi
\makeatother

\usepackage{color}
\usepackage{fancyvrb}
\newcommand{\VerbBar}{|}
\newcommand{\VERB}{\Verb[commandchars=\\\{\}]}
\DefineVerbatimEnvironment{Highlighting}{Verbatim}{commandchars=\\\{\}}
% Add ',fontsize=\small' for more characters per line
\usepackage{framed}
\definecolor{shadecolor}{RGB}{241,243,245}
\newenvironment{Shaded}{\begin{snugshade}}{\end{snugshade}}
\newcommand{\AlertTok}[1]{\textcolor[rgb]{0.68,0.00,0.00}{#1}}
\newcommand{\AnnotationTok}[1]{\textcolor[rgb]{0.37,0.37,0.37}{#1}}
\newcommand{\AttributeTok}[1]{\textcolor[rgb]{0.40,0.45,0.13}{#1}}
\newcommand{\BaseNTok}[1]{\textcolor[rgb]{0.68,0.00,0.00}{#1}}
\newcommand{\BuiltInTok}[1]{\textcolor[rgb]{0.00,0.23,0.31}{#1}}
\newcommand{\CharTok}[1]{\textcolor[rgb]{0.13,0.47,0.30}{#1}}
\newcommand{\CommentTok}[1]{\textcolor[rgb]{0.37,0.37,0.37}{#1}}
\newcommand{\CommentVarTok}[1]{\textcolor[rgb]{0.37,0.37,0.37}{\textit{#1}}}
\newcommand{\ConstantTok}[1]{\textcolor[rgb]{0.56,0.35,0.01}{#1}}
\newcommand{\ControlFlowTok}[1]{\textcolor[rgb]{0.00,0.23,0.31}{\textbf{#1}}}
\newcommand{\DataTypeTok}[1]{\textcolor[rgb]{0.68,0.00,0.00}{#1}}
\newcommand{\DecValTok}[1]{\textcolor[rgb]{0.68,0.00,0.00}{#1}}
\newcommand{\DocumentationTok}[1]{\textcolor[rgb]{0.37,0.37,0.37}{\textit{#1}}}
\newcommand{\ErrorTok}[1]{\textcolor[rgb]{0.68,0.00,0.00}{#1}}
\newcommand{\ExtensionTok}[1]{\textcolor[rgb]{0.00,0.23,0.31}{#1}}
\newcommand{\FloatTok}[1]{\textcolor[rgb]{0.68,0.00,0.00}{#1}}
\newcommand{\FunctionTok}[1]{\textcolor[rgb]{0.28,0.35,0.67}{#1}}
\newcommand{\ImportTok}[1]{\textcolor[rgb]{0.00,0.46,0.62}{#1}}
\newcommand{\InformationTok}[1]{\textcolor[rgb]{0.37,0.37,0.37}{#1}}
\newcommand{\KeywordTok}[1]{\textcolor[rgb]{0.00,0.23,0.31}{\textbf{#1}}}
\newcommand{\NormalTok}[1]{\textcolor[rgb]{0.00,0.23,0.31}{#1}}
\newcommand{\OperatorTok}[1]{\textcolor[rgb]{0.37,0.37,0.37}{#1}}
\newcommand{\OtherTok}[1]{\textcolor[rgb]{0.00,0.23,0.31}{#1}}
\newcommand{\PreprocessorTok}[1]{\textcolor[rgb]{0.68,0.00,0.00}{#1}}
\newcommand{\RegionMarkerTok}[1]{\textcolor[rgb]{0.00,0.23,0.31}{#1}}
\newcommand{\SpecialCharTok}[1]{\textcolor[rgb]{0.37,0.37,0.37}{#1}}
\newcommand{\SpecialStringTok}[1]{\textcolor[rgb]{0.13,0.47,0.30}{#1}}
\newcommand{\StringTok}[1]{\textcolor[rgb]{0.13,0.47,0.30}{#1}}
\newcommand{\VariableTok}[1]{\textcolor[rgb]{0.07,0.07,0.07}{#1}}
\newcommand{\VerbatimStringTok}[1]{\textcolor[rgb]{0.13,0.47,0.30}{#1}}
\newcommand{\WarningTok}[1]{\textcolor[rgb]{0.37,0.37,0.37}{\textit{#1}}}

\providecommand{\tightlist}{%
  \setlength{\itemsep}{0pt}\setlength{\parskip}{0pt}}\usepackage{longtable,booktabs,array}
\usepackage{calc} % for calculating minipage widths
% Correct order of tables after \paragraph or \subparagraph
\usepackage{etoolbox}
\makeatletter
\patchcmd\longtable{\par}{\if@noskipsec\mbox{}\fi\par}{}{}
\makeatother
% Allow footnotes in longtable head/foot
\IfFileExists{footnotehyper.sty}{\usepackage{footnotehyper}}{\usepackage{footnote}}
\makesavenoteenv{longtable}
\usepackage{graphicx}
\makeatletter
\newsavebox\pandoc@box
\newcommand*\pandocbounded[1]{% scales image to fit in text height/width
  \sbox\pandoc@box{#1}%
  \Gscale@div\@tempa{\textheight}{\dimexpr\ht\pandoc@box+\dp\pandoc@box\relax}%
  \Gscale@div\@tempb{\linewidth}{\wd\pandoc@box}%
  \ifdim\@tempb\p@<\@tempa\p@\let\@tempa\@tempb\fi% select the smaller of both
  \ifdim\@tempa\p@<\p@\scalebox{\@tempa}{\usebox\pandoc@box}%
  \else\usebox{\pandoc@box}%
  \fi%
}
% Set default figure placement to htbp
\def\fps@figure{htbp}
\makeatother
% definitions for citeproc citations
\NewDocumentCommand\citeproctext{}{}
\NewDocumentCommand\citeproc{mm}{%
  \begingroup\def\citeproctext{#2}\cite{#1}\endgroup}
\makeatletter
 % allow citations to break across lines
 \let\@cite@ofmt\@firstofone
 % avoid brackets around text for \cite:
 \def\@biblabel#1{}
 \def\@cite#1#2{{#1\if@tempswa , #2\fi}}
\makeatother
\newlength{\cslhangindent}
\setlength{\cslhangindent}{1.5em}
\newlength{\csllabelwidth}
\setlength{\csllabelwidth}{3em}
\newenvironment{CSLReferences}[2] % #1 hanging-indent, #2 entry-spacing
 {\begin{list}{}{%
  \setlength{\itemindent}{0pt}
  \setlength{\leftmargin}{0pt}
  \setlength{\parsep}{0pt}
  % turn on hanging indent if param 1 is 1
  \ifodd #1
   \setlength{\leftmargin}{\cslhangindent}
   \setlength{\itemindent}{-1\cslhangindent}
  \fi
  % set entry spacing
  \setlength{\itemsep}{#2\baselineskip}}}
 {\end{list}}
\usepackage{calc}
\newcommand{\CSLBlock}[1]{\hfill\break\parbox[t]{\linewidth}{\strut\ignorespaces#1\strut}}
\newcommand{\CSLLeftMargin}[1]{\parbox[t]{\csllabelwidth}{\strut#1\strut}}
\newcommand{\CSLRightInline}[1]{\parbox[t]{\linewidth - \csllabelwidth}{\strut#1\strut}}
\newcommand{\CSLIndent}[1]{\hspace{\cslhangindent}#1}

\KOMAoption{captions}{tableheading}
\makeatletter
\@ifpackageloaded{bookmark}{}{\usepackage{bookmark}}
\makeatother
\makeatletter
\@ifpackageloaded{caption}{}{\usepackage{caption}}
\AtBeginDocument{%
\ifdefined\contentsname
  \renewcommand*\contentsname{Table of contents}
\else
  \newcommand\contentsname{Table of contents}
\fi
\ifdefined\listfigurename
  \renewcommand*\listfigurename{List of Figures}
\else
  \newcommand\listfigurename{List of Figures}
\fi
\ifdefined\listtablename
  \renewcommand*\listtablename{List of Tables}
\else
  \newcommand\listtablename{List of Tables}
\fi
\ifdefined\figurename
  \renewcommand*\figurename{Figure}
\else
  \newcommand\figurename{Figure}
\fi
\ifdefined\tablename
  \renewcommand*\tablename{Table}
\else
  \newcommand\tablename{Table}
\fi
}
\@ifpackageloaded{float}{}{\usepackage{float}}
\floatstyle{ruled}
\@ifundefined{c@chapter}{\newfloat{codelisting}{h}{lop}}{\newfloat{codelisting}{h}{lop}[chapter]}
\floatname{codelisting}{Listing}
\newcommand*\listoflistings{\listof{codelisting}{List of Listings}}
\makeatother
\makeatletter
\makeatother
\makeatletter
\@ifpackageloaded{caption}{}{\usepackage{caption}}
\@ifpackageloaded{subcaption}{}{\usepackage{subcaption}}
\makeatother

\usepackage{bookmark}

\IfFileExists{xurl.sty}{\usepackage{xurl}}{} % add URL line breaks if available
\urlstyle{same} % disable monospaced font for URLs
\hypersetup{
  pdftitle={The Designometricon},
  pdfauthor={Schmettow},
  colorlinks=true,
  linkcolor={blue},
  filecolor={Maroon},
  citecolor={Blue},
  urlcolor={Blue},
  pdfcreator={LaTeX via pandoc}}


\title{The Designometricon}
\author{Schmettow}
\date{2026-07-02}

\begin{document}
\maketitle

\renewcommand*\contentsname{Table of contents}
{
\hypersetup{linkcolor=}
\setcounter{tocdepth}{2}
\tableofcontents
}

\bookmarksetup{startatroot}

\chapter*{Preface}\label{preface}
\addcontentsline{toc}{chapter}{Preface}

\markboth{Preface}{Preface}

This book is (becoming) a collection of articles about designometric
construction of research instruments.

To start with a hard-to-deny statement:

\begin{Shaded}
\begin{Highlighting}[]
\DecValTok{1} \SpecialCharTok{+} \DecValTok{1}
\end{Highlighting}
\end{Shaded}

\begin{verbatim}
[1] 2
\end{verbatim}

The discovery of designometrics started with the review of a paper,
which constructed a set of rating scales for comparing school chairs by
ergonomics criteria. The authors followed the psychometric workflow
neatly and collected data from a substantial sample of participants.
There was only one catch: They all sat on one and same chair! When the
purpose of the instrument is to rank a set of chairs, there is no way
this capacity can be evaluated using only one chair.

\bookmarksetup{startatroot}

\chapter{Introduction}\label{introduction}

It's the cube, stupid!

\bookmarksetup{startatroot}

\chapter{model \textless- M\_1\_design}\label{model---m_1_design}

\begin{Shaded}
\begin{Highlighting}[]
\FunctionTok{library}\NormalTok{(tidyverse)}
\FunctionTok{library}\NormalTok{(psych)}
\FunctionTok{library}\NormalTok{(mascutils)}
\FunctionTok{library}\NormalTok{(printr)}
\CommentTok{\#library(lavaan)}
\FunctionTok{options}\NormalTok{(}\AttributeTok{mc.cores =} \DecValTok{8}\NormalTok{)}

\NormalTok{purp.analysis }\OtherTok{\textless{}{-}}\NormalTok{ T}
\end{Highlighting}
\end{Shaded}

\begin{Shaded}
\begin{Highlighting}[]
\CommentTok{\#\textquotesingle{} Psychometric/designometric response matrix from long designometric data}
\CommentTok{\#\textquotesingle{} }
\CommentTok{\#\textquotesingle{} @param x long designometric data}
\CommentTok{\#\textquotesingle{} @returns psychometric response matrix.}
\CommentTok{\#\textquotesingle{} @examples}
\CommentTok{\#\textquotesingle{} ldmx \textless{}{-} expand.grid(Design = 1:7, Part = 1:5, Item = 1:3)}
\CommentTok{\#\textquotesingle{} ldmx$response \textless{}{-} rbeta(105, 2, 2)}
\CommentTok{\#\textquotesingle{} rm\_pmx(ldmx)}
\CommentTok{\#\textquotesingle{} rm\_dmx(ldmx)}

\NormalTok{rm\_pmx }\OtherTok{\textless{}{-}} \ControlFlowTok{function}\NormalTok{(x)}
\NormalTok{  x }\SpecialCharTok{|\textgreater{}} 
  \FunctionTok{group\_by}\NormalTok{(Part, Item) }\SpecialCharTok{|\textgreater{}} 
  \FunctionTok{summarize}\NormalTok{(}\AttributeTok{mean\_resp =} \FunctionTok{mean}\NormalTok{(response)) }\SpecialCharTok{|\textgreater{}} 
  \FunctionTok{ungroup}\NormalTok{() }\SpecialCharTok{|\textgreater{}} 
  \FunctionTok{arrange}\NormalTok{(Item) }\SpecialCharTok{|\textgreater{}} 
  \FunctionTok{spread}\NormalTok{(Item, }\AttributeTok{value =}\NormalTok{ mean\_resp) }\SpecialCharTok{|\textgreater{}} 
  \FunctionTok{select}\NormalTok{(}\SpecialCharTok{{-}}\NormalTok{Part)}

\CommentTok{\#\textquotesingle{} @rdname rm\_pmx}
\NormalTok{rm\_dmx }\OtherTok{\textless{}{-}} \ControlFlowTok{function}\NormalTok{(x) }
\NormalTok{  x }\SpecialCharTok{|\textgreater{}} 
  \FunctionTok{group\_by}\NormalTok{(Design, Item) }\SpecialCharTok{|\textgreater{}}
  \FunctionTok{summarize}\NormalTok{(}\AttributeTok{mean\_resp =} \FunctionTok{mean}\NormalTok{(response)) }\SpecialCharTok{|\textgreater{}} 
  \FunctionTok{ungroup}\NormalTok{() }\SpecialCharTok{|\textgreater{}} 
  \FunctionTok{spread}\NormalTok{(Item, }\AttributeTok{value =}\NormalTok{ mean\_resp) }\SpecialCharTok{|\textgreater{}} 
  \FunctionTok{select}\NormalTok{(}\SpecialCharTok{{-}}\NormalTok{Design)}


\CommentTok{\#\textquotesingle{} Psychometric function wrappers}
\CommentTok{\#\textquotesingle{} }
\CommentTok{\#\textquotesingle{} @param Data long designometric data}
\CommentTok{\#\textquotesingle{} @returns Results per Perspective and Scale}

\NormalTok{alpha\_ci }\OtherTok{\textless{}{-}} \ControlFlowTok{function}\NormalTok{(Data)\{}
\NormalTok{  Scale }\OtherTok{\textless{}{-}} \FunctionTok{str\_c}\NormalTok{(}\FunctionTok{distinct}\NormalTok{(Data, Scale)}\SpecialCharTok{$}\NormalTok{Scale) }
\NormalTok{  model\_psych }\OtherTok{\textless{}{-}} 
\NormalTok{    psych}\SpecialCharTok{::}\FunctionTok{alpha}\NormalTok{(}\FunctionTok{rm\_pmx}\NormalTok{(Data), }\AttributeTok{check.keys =} \ConstantTok{FALSE}\NormalTok{, }\AttributeTok{n.iter =} \DecValTok{100}\NormalTok{)}\SpecialCharTok{$}\NormalTok{boot }\SpecialCharTok{|\textgreater{}} 
    \FunctionTok{as\_tibble}\NormalTok{() }\SpecialCharTok{|\textgreater{}} 
    \FunctionTok{mutate}\NormalTok{(}\AttributeTok{Perspective =} \StringTok{"psychometric"}\NormalTok{)}
\NormalTok{  model\_design }\OtherTok{\textless{}{-}} 
\NormalTok{    psych}\SpecialCharTok{::}\FunctionTok{alpha}\NormalTok{(}\FunctionTok{rm\_dmx}\NormalTok{(Data), }\AttributeTok{check.keys =} \ConstantTok{FALSE}\NormalTok{, }\AttributeTok{n.iter =} \DecValTok{100}\NormalTok{)}\SpecialCharTok{$}\NormalTok{boot }\SpecialCharTok{|\textgreater{}} 
    \FunctionTok{as\_tibble}\NormalTok{() }\SpecialCharTok{|\textgreater{}} 
    \FunctionTok{mutate}\NormalTok{(}\AttributeTok{Perspective =} \StringTok{"designometric"}\NormalTok{)}
\NormalTok{  out }\OtherTok{\textless{}{-}} 
    \FunctionTok{bind\_rows}\NormalTok{(model\_psych,}
\NormalTok{              model\_design) }\SpecialCharTok{|\textgreater{}} 
    \FunctionTok{select}\NormalTok{(Perspective, std.alpha) }\SpecialCharTok{|\textgreater{}} 
    \FunctionTok{group\_by}\NormalTok{(Perspective) }\SpecialCharTok{|\textgreater{}} 
    \FunctionTok{summarize}\NormalTok{(}\AttributeTok{center =} \FunctionTok{mean}\NormalTok{(std.alpha),}
              \AttributeTok{lower =} \FunctionTok{quantile}\NormalTok{(std.alpha, .}\DecValTok{025}\NormalTok{),}
              \AttributeTok{upper =} \FunctionTok{quantile}\NormalTok{(std.alpha, .}\DecValTok{975}\NormalTok{)) }\SpecialCharTok{|\textgreater{}} 
    \FunctionTok{mutate}\NormalTok{(}\AttributeTok{Scale =}\NormalTok{ Scale) }\SpecialCharTok{|\textgreater{}} 
\NormalTok{    mascutils}\SpecialCharTok{::}\FunctionTok{go\_first}\NormalTok{(Scale, Perspective)}
\NormalTok{  out}
\NormalTok{\}}

\NormalTok{item\_rel }\OtherTok{\textless{}{-}} \ControlFlowTok{function}\NormalTok{(Data)\{}
  \CommentTok{\#Data \textless{}{-} D\_1 |\textgreater{} filter(Scale == "HQI")}
\NormalTok{  Scale }\OtherTok{\textless{}{-}} \FunctionTok{str\_c}\NormalTok{(}\FunctionTok{distinct}\NormalTok{(Data, Scale)}\SpecialCharTok{$}\NormalTok{Scale)}
\NormalTok{  model\_psych }\OtherTok{\textless{}{-}} 
\NormalTok{    psych}\SpecialCharTok{::}\FunctionTok{alpha}\NormalTok{(}\FunctionTok{rm\_pmx}\NormalTok{(Data), }\AttributeTok{check.keys =} \ConstantTok{FALSE}\NormalTok{)}\SpecialCharTok{$}\NormalTok{item.stats }\SpecialCharTok{|\textgreater{}} 
    \FunctionTok{as\_tibble}\NormalTok{(}\AttributeTok{rownames =} \StringTok{"Item"}\NormalTok{) }\SpecialCharTok{|\textgreater{}} 
    \FunctionTok{mutate}\NormalTok{(}\AttributeTok{Perspective =} \StringTok{"psychometric"}\NormalTok{)}
\NormalTok{  model\_design }\OtherTok{\textless{}{-}} 
\NormalTok{    psych}\SpecialCharTok{::}\FunctionTok{alpha}\NormalTok{(}\FunctionTok{rm\_dmx}\NormalTok{(Data), }\AttributeTok{check.keys =} \ConstantTok{FALSE}\NormalTok{)}\SpecialCharTok{$}\NormalTok{item.stats }\SpecialCharTok{|\textgreater{}} 
    \FunctionTok{as\_tibble}\NormalTok{(}\AttributeTok{rownames =} \StringTok{"Item"}\NormalTok{) }\SpecialCharTok{|\textgreater{}} 
    \FunctionTok{mutate}\NormalTok{(}\AttributeTok{Perspective =} \StringTok{"designometric"}\NormalTok{)}
  
\NormalTok{  out }\OtherTok{\textless{}{-}} 
    \FunctionTok{bind\_rows}\NormalTok{(model\_psych,}
\NormalTok{              model\_design) }\SpecialCharTok{|\textgreater{}} 
    \FunctionTok{mutate}\NormalTok{(}\AttributeTok{Scale =}\NormalTok{ Scale) }\SpecialCharTok{|\textgreater{}} 
    \FunctionTok{go\_first}\NormalTok{(Scale, Item, Perspective) }\SpecialCharTok{|\textgreater{}} 
    \FunctionTok{arrange}\NormalTok{(Scale, Item, Perspective)}
\NormalTok{  out}
\NormalTok{\}}

\NormalTok{parallel\_analysis }\OtherTok{\textless{}{-}} \ControlFlowTok{function}\NormalTok{(Data, n, persp, scales)\{}
  \ControlFlowTok{if}\NormalTok{ (persp }\SpecialCharTok{==} \StringTok{"D"}\NormalTok{) \{}
\NormalTok{    data }\OtherTok{\textless{}{-}} \FunctionTok{rm\_dmx}\NormalTok{(Data)}
\NormalTok{    main }\OtherTok{\textless{}{-}} \FunctionTok{str\_c}\NormalTok{(}\StringTok{"Designometric Parallel Analysis of "}\NormalTok{, scales)}
\NormalTok{  \}}
  \ControlFlowTok{if}\NormalTok{ (persp }\SpecialCharTok{==} \StringTok{"P"}\NormalTok{) \{}
\NormalTok{    data }\OtherTok{\textless{}{-}} \FunctionTok{rm\_pmx}\NormalTok{(Data)}
\NormalTok{    main }\OtherTok{\textless{}{-}} \FunctionTok{str\_c}\NormalTok{(}\StringTok{"Psychometric Parallel Analysis of "}\NormalTok{, scales)}
\NormalTok{  \}}
\NormalTok{  psych}\SpecialCharTok{::}\FunctionTok{fa.parallel}\NormalTok{(data,}
                   \AttributeTok{fa =} \StringTok{"fa"}\NormalTok{,}
                   \AttributeTok{fm =} \StringTok{"minres"}\NormalTok{,}
                   \AttributeTok{nfactors=}\NormalTok{n,}
                   \AttributeTok{main=}\NormalTok{main)}
    
\NormalTok{\}}
\end{Highlighting}
\end{Shaded}

\bookmarksetup{startatroot}

\chapter{The designometric perspective and the psychometric
fallacy}\label{the-designometric-perspective-and-the-psychometric-fallacy}

\section{Introduction}\label{introduction-1}

In modern industrial practice, rating scales have their place as an
always available and cheap method for comparing or benchmarking designs.
In the decision process, the everyday value of a rating scales stands
and falls with two properties: validity and reliability.

\emph{Developing} a valid and reliable rating scale is quite an
undertaking. \emph{Psychometrics} is the science of assigning numbers to
persons so that they can be compared by psychological functioning.
Traditionally, this served to measure skills, such as mathematical
intelligence or comprehension of language. With the time, researchers
also became more interested in elusive properties of persons, such as
psycho-social tendencies (known as the Big Five).

After the landrush phase of the UX revolution, Bargas-Avila \& Hoernbaek
((Old wine in new bottles)) counted hundreds of new rating scale
instruments. Barely any of these instrument underwent the same scrutiny
as, for example, a rating scale for psycho-pathological diagnosis would
have to. But, frequently some psychometric tools were used at one point
during development time, for example by reporting reliability estimates.
Also for users of such instruments, it is common to perform basic
psychometric sanity checks on their data. The central point of this
paper is a certain catch that can occur when translating between
psychological research and design research, which we call the
\emph{psychometric fallacy}.

In psychometric situations, the atomic observation is an encounter of a
person with a test item. This is repeated with more items to improve the
precision of the measurement. If many persons are assessed this way, the
result is a Person X Item \emph{response matrix}, from which
\emph{person scores} can be extracted for ranking. \emph{Item scores},
are mostly used during scale development.

The logical argument developed in this paper is that instruments in
design research exist to rank designs and the atomic observation is an
encounter of a design with a person and an item. This forms a box of
data, which can be collapsed by three perspectives. We demonstrate how
standard psychometric tools can be used correctly using the
designometric response matrix. We call it the psychometric fallacy, when
a psychometric matrix is used instead.

\ldots{}

Aim of this study is also to seek empirical evidence that the
psychometric fallacy is not just sophistry, but can result in real
biases when developing or using rating scales. For this purpose, data
from five experiments was subjected to typical rating scale validation
techniques under both perspectives, psychometric (pretending the
fallacy) and designometric (using the proper response matrix).

\subsection{Psychometrics}\label{psychometrics}

When it only takes a single probe to accurately estimate the body
temperature of a person, why are psychological tests and math exams
composed of several items? The primarily reason is that psychological
instruments are highly noisy, be it self-report scales, reaction times
or physiological measures. The idea of repeated measures goes back to
\emph{Classic Test Theory}, which solves this problem by decomposing
every measure into a relevant \emph{systematic component} and a nuisance
\emph{error component}. By definition, the systematic component
re-occurs and is strengend with every measurement, whereas the error is
fully random, cancelling itself out in the long run. When certain
conditions are met, it is possible to create an arbitrary precise
estimate by adding more repetitions.

The primary practical uses of psychometric instruments are either about
making cost effective \emph{prediction} about a person. A good score in
a written drivers test predicts that a person will make fewer errors on
the road which decreases the chance of collisions. Psycho-diagnostic
rating scales are often used in screening tests, e.g.~for depression,
whereas performance tests are used in competitive situations, such as
personnel selection or education.

\subsubsection{Item selection}\label{item-selection}

The first challenge when designing a psychometric inventory is to
understand the domain, ideally in terms of involved areas of
psychological functioning. Based on the domain analysis, the researcher
creates a candidate \emph{item pool}, which is usually much larger than
the targeted set of items. Up to this point, the process is mostly
qualitative, divergent and creative.

Several psychometric methods can be used to successively reduce the
initial set of items to reliable scales and to find effective
multi-scale strucures.

According to CTT, the errors of multiple items cancel each other out,
\emph{if they fully agree} on the systematic component. This can be the
case with extremely repetitive items, such as trials in the Stroop task.
But in practice, item scores only agree to some degree on what they
measure. Scale reliability by Cronbach \(\alpha\) is a statistic to
measure the overall level of agreement. From this derives a basic
procedure for item selection, where reliability of the full item set is
compared to the data set \emph{excluding} item \(i\). If scale
reliability improves by removing the item, the item is marked for
removal.

The procedure of step-wise removal is sometimes adequate, but can also
fail, when the alleged property truly has more than one component. In
this case, \emph{factor-analytic} methods can be used to separate
components into multiple sub scales with good consistency.

\subsubsection{Factor Structures}\label{factor-structures}

In modern psychometrics, the relation between a properties of persons
and multiple measure is known as the distinction between \emph{latent
variables} (not observable, but true) and \emph{indicator variables}
(observable, but imperfect).

Modern instruments often carry a complex domain structure, where
multiple latent variables are put in relation to each other. A prominent
example is the Big 5 inventory, which claims that social behaviour can
be predicted by five psycho-social traits, each of which is assessed by
a main scale divided into multi-item subscales. For the five main traits
a high level of independence can be expected by the way the Big5 was
constructed, but the subscales should correlate more strongly.

While the primary aim of domain analysis is to gain sufficient coverage
of the domain, it often at least indicates a possible structure for a
multi-item instrument. In some cases, a domain analysis is driven by a
theoretical structure in the first place. For example, a rating scale
for mental workload could very well be based on the multiple-resource
theory, which predicts that different sensory modes are processed
independently, translating into one scale per sensory mode.

\begin{enumerate}
\def\labelenumi{\arabic{enumi}.}
\tightlist
\item
  Auditory load: \emph{It is difficult to understand the signal tones.}
\item
  Visual load: \emph{The amount of visual information can be
  overwhelming}.
\end{enumerate}

Prior structures often also emerge from qualitative results in
mixed-method psychometric processes. \ldots{}

When prior structures exist, Confirmatory Factor Analysis (CFA) is the
most recommended technique to critically assess prior assumptions about
the latent variable structure. In the above example of the Big5
inventory, a hierarchical CFA model can be used to verify that main
scales correlate weakly and subscales correlate moderately. Once a good
model is established, CFA also supercedes Cronbach \(\alpha\), as
parameter loadings can be used to diagnose items.

When a robust prior structure is not available, a common technique for
finding new structures is Exploratory Factor Analysis, which requires
the researcher to name the number of factors and how these factors are
correlated. Several methods have been proposed to identify the
\emph{optimal number of factors} before-hand. Early tools used the
decline in Kaiser eigenvalues and the visual \emph{elbow} criterion,
whereas modern tools use resampling techniques to determinje the optimal
number of factors. The second choice to make is \emph{scale rotation.}
Orthogonal rotation applies, when the components are largely
independent, such as arithmetic skills and text comprehension. For true
sub scales a significant correlations between underlying factors is very
likely, and oblique rotation should be used, instead.

\subsubsection{The Response Matrix}\label{the-response-matrix}

All psychometric methods mentioned so far have in common that measures
are provided in form of response matrices, where rows identify with
participants and columns with items. A common operation during item
selection is to estimate consistency on the full response matrix, remove
a dubious item from the matrix to see if this improves consistency. EFA
takes a response matrix as input and produces item loadings for it. For
multi-scale data CFA tools take response matrix with named columns as
input, together with a formula to group items to factors. \ldots{}

The idea of response matrices is most pronounced in item response theory
(IRT), a psychometric approach rivaling classic test theory and factors
analysis. In IRT, each cell in the response matrix is seen as an
encounter between an item and a participant, from which a measure
arises. Items, like participants, are seen as individuals, with
different response characteristics during the encounter. In the most
basic Rasch model, the probability of success in a person-task encounter
depends only on the difference between task difficulty and the person
ability. More complex IRT models allow more subtle item characteristics,
and sophisticated tools have been developed to detect and prevent
\emph{differential item functioning}

in world-spanning studies like TIMMS and PISA. IRT models take the
person x item matrix as input and produce estimates on both margins,
item parameters and person parameters. During the development of an IRT
model, sophisticated tools are used to select only those items that
strictly behave as the model states.

For all levels of sophistication or schools of thought, what counts
during scale development is that item of a test are well-behaved. In the
same way teachers use multiple tests to estimate a student's skill with
adequate accuracy, many responses are needed to precisely assess the
characteristics of an item during development time. This is what creates
the demand for the enormous sample sizes, for which psychometric
development is most feared for.

\subsubsection{Sample sizes}\label{sample-sizes}

The enormous sample sizes consumed during psychometric scale validation
have several causes:

\begin{itemize}
\tightlist
\item
  Psychometric techniques tend to have many parameters and by the rule
  that the number of independent observations must at least match the
  number of parameters, a lower boundary can be established. A 24 item
  three-way EFA already produces 72 parameters. But, during scale
  validation, the number of items is often three times larger, which
  adds up.
\item
  With noisy data certain estimation procedures need a lot of data to
  converge properly, for example in conditional maximum likelihood
  estimation.
\item
  When a structure is desired, but no prior structure exists, EFA can be
  used to find factors. But the consequence is that the data set is
  consumed. To confirm the found structure by CFA, a new sample is
  required.
\item
  The main areas of psychometric research are education and clinical
  diagnosis, where a rating scale or test can have a significant impact
  on a person's life. These instruments ought to be extra-hardened
  against biases and require the greatest scrunity. For item in a PISA
  test, proof is required that it has the same characteristics all
  around the globe.
\end{itemize}

\subsection{Designometrics}\label{designometrics}

Everything said above is essentially rooted in mathematical theory, and
therefore independent of semantics. Formally, constructing rating scales
to rank humans should be not much different to constructing a usability
metric for websites. When a UX scale is used to decide between designs A
and B in industrial practice, it is legit to ask how precise the
measurements are. Indeed, designometric scales can efficiently be
constructed and evaluated using existing psychometric tools, such as
Cronbach \(\alpha\), factor analysis techniques ans IRT models for
maximum rigor.

There are just two differences: First, in psychometrics the entity to be
ranked is persons, whereas designometric applications one or more
designs get numbers attached. Second, a psychometric measure is an
encounter of a person with an item, whereas a designometric measure is
an encounter of a design with a person and an item. The three-way
encounter and designs being subject to measurement is what defines
\emph{designometric situations}.

\subsubsection{The designometric
perspective}\label{the-designometric-perspective}

A complete designometric data set is a cross-product of three samples,
designs, users and items. It forms a ``response box'', rather than a
matrix, and can not directly be processed with psychometric tools.

A practical solution is to average across one factor, which produces a
flat response matrix to fit into psychometric tools. For designometric
instrument validation, averaging across Person creates a design-by-item
\emph{designometric response matrix}. This matrix takes the place of the
psychometric response matrix and researchers can use it with standard
psychometric tools to select proper items and find good structures.

\subsubsection{Designometric sampling}\label{designometric-sampling}

During psychometric scale development, This also implies that for a
proper item selection and factorizing process a substantial sample of
designs is required. This can be a huge problem, depending on the class
of designs. For e-government websites a large design sample will be
easier to come by than a sample of human-like robots or self-driving
cars.

When Design takes the place of Person, it is implied that items must now
be well-behaved in ranking designs, and substantial samples of designs
are now required to prove that. This can be a minor or huge problem,
depending on the class of designs. For e-government websites a large
design sample will be easier to come by than a sample of human-like
robots or self-driving cars.

A connected problem is that the designometric perspective
participant-level information is reduced, but they are still involved in
the designometric encounter. In the special case that scales are used
for first impression ratings, it is possible to rund cross-product
complete designometric encounters. For example, when we used the
Eeriness scale on artificial faces, a single observation only took
around 4 seconds, making it relatively easy to fill the designometric
box with data, even repeatedly ((BT Robbin Koopmans)).

When a real interactive experience is subject of the measure, a
measurent can take from several minutes to hours and a complete
designometric encounter becomes impractical. A way to mitigate this
problem is to use an experimental design that is \emph{planned
incomplete}. Essentially, a planned incomplete validation study has all
participants encounter only a partition of the design sample. For
example a study consisting of a sample of 100 designs let every
participant encounter a different set of ten designs. As long as all
designs are covered by at least one participant, this will result in a
complete design-by-item matrix after collapsing along participants.

A variation of planned incomplete studies is to \emph{successively}
build the sample of designs. This is especially useful, when dealing
with emerging classes of designs. This happened in the BUS-11 studies,
where initially it was difficult to build a substantial sample, before
it became more common.

\subsubsection{The psychometric fallacy}\label{the-psychometric-fallacy}

Designometric scales can be developed and validated with established
psychometric tools, when using proper design x item response matrices
and a sufficiently large sample of designs to prove that items are well
behaved. In contrast, many designometric instruments have not been
validated using a large sample of designs, but rather on a psychometric
matrix. This we call the \emph{psychometric fallacy}.

Formally, a designometric box can produce a psychometric response matrix
by averaging over Design. When a scale validation study in design
research is under the psychometric fallacy, validation metrics such as
item reliability may be meaningless for the purpose of ranking designs.
Rather, the metric will refer to the capability of the item to
discriminate persons by their sensitivity to the design feature in
question. For example, a scale for comparing designs by beauty would
become a scale to rank persons by how critical they are with respect to
interface beauty. This is not the same and in the next section we show
by simulation that the differences between designometric and
psychometric perspectives can be dramatic.

During the scale development process, the psychometric fallacy appears
in two forms, one is repairable, whereas the other is fatal. We speak of
a repairable psychometric fallacy, when the validation study properly
collected designometric data, but used psychometric response matrices
for validation. This can be repaired by redoing the analysis now using a
designometric matrix.

A study that did not collect a sample of designs, but used only a single
or very few designs fell for the \emph{fatal psychometric fallacy}. In
these cases, researchers have failed to recognize a simple truth: The
capacity to discriminate between designs can impossibly be validated on
a single design. Every alleged designometric instrument, where this has
happened during the validation phase, cannot be trusted.

Recall that psychometric validations require large samples of
participants! Similar sample sizes will be required for designs when
validating designometric instruments. So, even validation studies that
included multiple designs, may in practice not be repairable, because
the sample of designs is too small for that type of analysis
(e.g.~factor analysis). To give an example MacDorman (XY) validated the
Godspeed Index, a common inventory to evaluate robotic designs. While
scale reliability was assessed under a psychometric perspective,
validity was correctly assessed using a sample of four designs. This
allowed the authors to do a rough test on the ranking capabilities of
their scales, but would not be suitable assess more detailed metric
properties of the scale.

As a milder form \emph{run-time psychometric fallacy} appears when an
instrument is used in practice to take measures on a single design. The
result will inevitable look like a psychometric response matrix and,
given that publication rules (e.g.~APA guidelines) often require to
report some psychometric properties, it may be tempting for the
researcher to run a psychometric test on reliability. While the run-time
fallacy does not have the same impact as development-time fallacies, it
may have cause some head aches when a validated instrument seems to have
poor reliability.

In the following section, we construct a case by simulation, showing
that psychometric and designometric perspectives can result in
dramatically different results. In the remainder of the study, we will
use data from past experiments to evaluate how strong the deviations are
with several existing and commonly used rating scales.

The formal argumeńt for the psychometric fallacy is that using
psychometric response matrices would allow to construct an instrument
for ranking designs using a single design.

The aim of the present study is to assess the possible risks and
consequences of falsely using the psychometric perspective. In the
following section, we construct a case by data simulation, where
psychometric and designometric perspectives result are dramatically
different. In the remainder of the study, we will use data from past
experiments to compare psychometric versus designometric perspective on
several widely used UX rating scales.

\section{Simulation study}\label{simulation-study}

The following example demonstrates the difference by simulating an
extreme situation, where a fictional three-item scale for Coolness is
highly reliable for persons, but has no reliability at all for
discerning the tested designs. Such a pattern can occur for the trivial
reason that the sample have little or no variance with respect to
Coolness. In the following simulation, we assume that the Coolness scale
be tested on a sample of 50 designs and 50 participants. The key here is
that participants vary strongly in their appreciation of Coolness
(\(\sigma_\textrm{Part} = .2\)), whereas the sample of designs varies
little in Coolness (\(\sigma_\textrm{Design} = .02\)), perhaps

\begin{Shaded}
\begin{Highlighting}[]
\FunctionTok{set.seed}\NormalTok{(}\DecValTok{42}\NormalTok{)}

\NormalTok{n\_Design }\OtherTok{=} \DecValTok{20}
\NormalTok{n\_Part   }\OtherTok{=} \DecValTok{20}
\NormalTok{n\_Item  }\OtherTok{=}  \DecValTok{4}
\NormalTok{n\_Obs }\OtherTok{=}\NormalTok{ n\_Design }\SpecialCharTok{*}\NormalTok{ n\_Part }\SpecialCharTok{*}\NormalTok{ n\_Item}

\NormalTok{Designs }\OtherTok{\textless{}{-}} \FunctionTok{tibble}\NormalTok{(}\AttributeTok{Design      =} \FunctionTok{as.factor}\NormalTok{(}\DecValTok{1}\SpecialCharTok{:}\NormalTok{n\_Design),}
                  \AttributeTok{cool\_Design =} \FunctionTok{rnorm}\NormalTok{(n\_Design, }\SpecialCharTok{{-}}\FloatTok{0.5}\NormalTok{, .}\DecValTok{02}\NormalTok{)) }\DocumentationTok{\#\# low level, little variation}

\NormalTok{Parts   }\OtherTok{\textless{}{-}} \FunctionTok{tibble}\NormalTok{(}\AttributeTok{Part        =} \FunctionTok{as.factor}\NormalTok{(}\DecValTok{1}\SpecialCharTok{:}\NormalTok{n\_Part),}
                  \AttributeTok{cool\_Part   =} \FunctionTok{rnorm}\NormalTok{(n\_Part, }\DecValTok{0}\NormalTok{, .}\DecValTok{2}\NormalTok{)) }\DocumentationTok{\#\# strong variance in tendency to judge sth. cool}

\NormalTok{Items   }\OtherTok{\textless{}{-}} \FunctionTok{tibble}\NormalTok{(}\AttributeTok{Scale       =} \StringTok{"Coolness"}\NormalTok{,}
                  \AttributeTok{Item        =} \FunctionTok{as.factor}\NormalTok{(}\DecValTok{1}\SpecialCharTok{:}\DecValTok{4}\NormalTok{),}
                  \AttributeTok{cool\_Item   =} \FunctionTok{rnorm}\NormalTok{(n\_Item,  }\DecValTok{0}\NormalTok{, .}\DecValTok{2}\NormalTok{)) }\DocumentationTok{\#\# item strength: understating items get lower values}

\NormalTok{Coolness     }\OtherTok{\textless{}{-}} \FunctionTok{expand\_grid}\NormalTok{(}\AttributeTok{Design =}\NormalTok{ Designs}\SpecialCharTok{$}\NormalTok{Design,}
                       \AttributeTok{Part   =}\NormalTok{ Parts}\SpecialCharTok{$}\NormalTok{Part,}
                       \AttributeTok{Item   =}\NormalTok{ Items}\SpecialCharTok{$}\NormalTok{Item) }\SpecialCharTok{|\textgreater{}} 
  \FunctionTok{left\_join}\NormalTok{(Designs) }\SpecialCharTok{|\textgreater{}} 
  \FunctionTok{left\_join}\NormalTok{(Parts) }\SpecialCharTok{|\textgreater{}} 
  \FunctionTok{left\_join}\NormalTok{(Items) }\SpecialCharTok{|\textgreater{}} 
  \FunctionTok{mutate}\NormalTok{(}\AttributeTok{response =}\NormalTok{ mascutils}\SpecialCharTok{::}\FunctionTok{rescale\_zero\_one}\NormalTok{(cool\_Design }\SpecialCharTok{+}\NormalTok{ cool\_Part }\SpecialCharTok{{-}}\NormalTok{ cool\_Item }\SpecialCharTok{+} \FunctionTok{rnorm}\NormalTok{(n\_Obs, }\DecValTok{0}\NormalTok{, .}\DecValTok{5}\NormalTok{)))}
\end{Highlighting}
\end{Shaded}

\begin{verbatim}
Joining with `by = join_by(Design)`
Joining with `by = join_by(Part)`
Joining with `by = join_by(Item)`
\end{verbatim}

\begin{Shaded}
\begin{Highlighting}[]
\FunctionTok{alpha\_ci}\NormalTok{(Coolness) }\SpecialCharTok{|\textgreater{}}\NormalTok{ knitr}\SpecialCharTok{::}\FunctionTok{kable}\NormalTok{()}
\end{Highlighting}
\end{Shaded}

\begin{verbatim}
`summarise()` has grouped output by 'Part'. You can override using the
`.groups` argument.
Number of categories should be increased in order to count frequencies.
`summarise()` has grouped output by 'Design'. You can override using the
`.groups` argument.
Number of categories should be increased in order to count frequencies.
\end{verbatim}

\begin{verbatim}
Warning in psych::alpha(rm_dmx(Data), check.keys = FALSE, n.iter = 100): Some items were negatively correlated with the first principal component and probably 
should be reversed.  
To do this, run the function again with the 'check.keys=TRUE' option
\end{verbatim}

\begin{verbatim}
Some items ( 1 ) were negatively correlated with the first principal component and 
probably should be reversed.  
To do this, run the function again with the 'check.keys=TRUE' option
\end{verbatim}

\begin{longtable}[]{@{}llrrr@{}}
\toprule\noalign{}
Scale & Perspective & center & lower & upper \\
\midrule\noalign{}
\endhead
\bottomrule\noalign{}
\endlastfoot
Coolness & designometric & 0.0745499 & -0.5345157 & 0.4836190 \\
Coolness & psychometric & 0.9284720 & 0.8743449 & 0.9586799 \\
\end{longtable}

This simple example demonstrate that a scale can produce excellent
reliability when measuring person sensitivity, but poor and uncertain
reliability on designs. Under the psychometric fallacy it can happen
that excellent reliability is reported, while it is actually unknown, or
very poor.

In the following study we use data from several previous experiments
that had produced designometric data sets.

\section{Methods}\label{methods}

From a theoretical perspective the psychometric fallacy is obvious and
we have demonstrated by simulation that the worst case is possible.
Here, we explore the biases that can occur when a psychometric response
matrix is used, in place of a designometric. Designometric data was
collected from several experiments to compare three commonly used
psychometric statistics under the correct designometric perspective and
under the psychometric fallacy.

\subsection{Data sets}\label{data-sets}

The data used for analysis originates from five experiments (DK, PS, AH,
QB, DN). While these experiments were carried out to test their own
hypotheses, they have in common that participants saw pictures of many
designs and were asked to respond to items taken from one or more
scales. In QB and DN participants saw pictures of home pages and
responded to several user experience scales, whereas in AH, DK and PS
the stimuli were robot faces. Some of the original experiments used
manipualtion of presentation time to collect data on subconscious
cognitive processing. For the analysis here, only responses at
presentation times of 500ms and 2000m were used.

Per trial participants saw a single design followed by a random single
item, resulting in a sparse designometric box. However, when collapsing
the box to either psychometric RM or designometric RM, the result is
completely filled response matrices.

\begin{Shaded}
\begin{Highlighting}[]
\FunctionTok{load}\NormalTok{(}\StringTok{"DMX\_data.Rda"}\NormalTok{)}
\end{Highlighting}
\end{Shaded}

\begin{Shaded}
\begin{Highlighting}[]
\NormalTok{D\_1 }\SpecialCharTok{|\textgreater{}} 
  \FunctionTok{ggplot}\NormalTok{(}\FunctionTok{aes}\NormalTok{(}\AttributeTok{x =}\NormalTok{ response)) }\SpecialCharTok{+}
  \FunctionTok{geom\_histogram}\NormalTok{() }\SpecialCharTok{+}
  \FunctionTok{facet\_wrap}\NormalTok{(}\SpecialCharTok{\textasciitilde{}}\NormalTok{Scale, }\AttributeTok{scale =} \StringTok{"free\_y"}\NormalTok{)}
\end{Highlighting}
\end{Shaded}

\begin{verbatim}
`stat_bin()` using `bins = 30`. Pick better value with `binwidth`.
\end{verbatim}

\pandocbounded{\includegraphics[keepaspectratio]{DMX_1_files/figure-pdf/unnamed-chunk-6-1.pdf}}

\begin{Shaded}
\begin{Highlighting}[]
\NormalTok{D\_1 }\SpecialCharTok{|\textgreater{}} 
  \FunctionTok{distinct}\NormalTok{(Study, Design) }\SpecialCharTok{|\textgreater{}} 
  \FunctionTok{group\_by}\NormalTok{(Study) }\SpecialCharTok{|\textgreater{}} 
  \FunctionTok{summarize}\NormalTok{(}\AttributeTok{n\_Design =} \FunctionTok{n}\NormalTok{()) }\SpecialCharTok{|\textgreater{}} 
  \FunctionTok{ungroup}\NormalTok{() }\SpecialCharTok{|\textgreater{}} 
  \FunctionTok{left\_join}\NormalTok{(D\_1 }\SpecialCharTok{|\textgreater{}} 
              \FunctionTok{distinct}\NormalTok{(Study, Part) }\SpecialCharTok{|\textgreater{}} 
              \FunctionTok{group\_by}\NormalTok{(Study) }\SpecialCharTok{|\textgreater{}} 
              \FunctionTok{summarize}\NormalTok{(}\AttributeTok{n\_Part =} \FunctionTok{n}\NormalTok{())}\SpecialCharTok{|\textgreater{}} 
              \FunctionTok{ungroup}\NormalTok{()) }\SpecialCharTok{|\textgreater{}} 
  \FunctionTok{left\_join}\NormalTok{(D\_1 }\SpecialCharTok{|\textgreater{}} 
              \FunctionTok{group\_by}\NormalTok{(Study) }\SpecialCharTok{|\textgreater{}} 
              \FunctionTok{summarize}\NormalTok{(}\AttributeTok{n\_Obs =} \FunctionTok{n}\NormalTok{())}\SpecialCharTok{|\textgreater{}} 
              \FunctionTok{ungroup}\NormalTok{()}
\NormalTok{  )  }\SpecialCharTok{|\textgreater{}}\NormalTok{ knitr}\SpecialCharTok{::}\FunctionTok{kable}\NormalTok{()}
\end{Highlighting}
\end{Shaded}

\begin{verbatim}
Joining with `by = join_by(Study)`
Joining with `by = join_by(Study)`
\end{verbatim}

\begin{longtable}[]{@{}lrrr@{}}
\toprule\noalign{}
Study & n\_Design & n\_Part & n\_Obs \\
\midrule\noalign{}
\endhead
\bottomrule\noalign{}
\endlastfoot
AH & 20 & 45 & 10800 \\
DK & 80 & 35 & 2800 \\
DN & 48 & 42 & 8064 \\
PS & 87 & 39 & 2808 \\
QB & 76 & 25 & 1900 \\
SP & 66 & 40 & 1440 \\
\end{longtable}

\begin{Shaded}
\begin{Highlighting}[]
\NormalTok{D\_1 }\SpecialCharTok{|\textgreater{}} 
  \FunctionTok{distinct}\NormalTok{(Scale, Design) }\SpecialCharTok{|\textgreater{}} 
  \FunctionTok{group\_by}\NormalTok{(Scale) }\SpecialCharTok{|\textgreater{}} 
  \FunctionTok{summarize}\NormalTok{(}\AttributeTok{n\_Design =} \FunctionTok{n}\NormalTok{()) }\SpecialCharTok{|\textgreater{}} 
  \FunctionTok{ungroup}\NormalTok{() }\SpecialCharTok{|\textgreater{}} 
  \FunctionTok{left\_join}\NormalTok{(D\_1 }\SpecialCharTok{|\textgreater{}} 
              \FunctionTok{distinct}\NormalTok{(Scale, Part) }\SpecialCharTok{|\textgreater{}} 
              \FunctionTok{group\_by}\NormalTok{(Scale) }\SpecialCharTok{|\textgreater{}} 
              \FunctionTok{summarize}\NormalTok{(}\AttributeTok{n\_Part =} \FunctionTok{n}\NormalTok{())}\SpecialCharTok{|\textgreater{}} 
              \FunctionTok{ungroup}\NormalTok{()) }\SpecialCharTok{|\textgreater{}} 
  \FunctionTok{left\_join}\NormalTok{(D\_1 }\SpecialCharTok{|\textgreater{}} 
              \FunctionTok{group\_by}\NormalTok{(Scale) }\SpecialCharTok{|\textgreater{}} 
              \FunctionTok{summarize}\NormalTok{(}\AttributeTok{n\_Obs =} \FunctionTok{n}\NormalTok{())}\SpecialCharTok{|\textgreater{}} 
              \FunctionTok{ungroup}\NormalTok{()}
\NormalTok{  )  }\SpecialCharTok{|\textgreater{}}\NormalTok{ knitr}\SpecialCharTok{::}\FunctionTok{kable}\NormalTok{()}
\end{Highlighting}
\end{Shaded}

\begin{verbatim}
Joining with `by = join_by(Scale)`
Joining with `by = join_by(Scale)`
\end{verbatim}

\begin{longtable}[]{@{}lrrr@{}}
\toprule\noalign{}
Scale & n\_Design & n\_Part & n\_Obs \\
\midrule\noalign{}
\endhead
\bottomrule\noalign{}
\endlastfoot
Attractiveness & 66 & 40 & 1440 \\
Beauty & 48 & 42 & 2688 \\
Credib & 76 & 25 & 500 \\
HQI & 76 & 25 & 700 \\
HQS & 76 & 25 & 700 \\
Hedonism & 48 & 42 & 2688 \\
Usability & 48 & 42 & 2688 \\
nEeriness & 127 & 119 & 16408 \\
\end{longtable}

\subsection{Scales}\label{scales}

For the following rating scales responses have been extracted from the
original experimental data:

The \emph{Eeriness} scale has been developed for measuring negative
emotional responses towards robot faces and is a primary research tool
on the Uncanny Valley phenomenon. Ho \& MacDorman(2017) present an
advanced psychometric validation of the scale. The study made use of 12
animated characters (Designs), avoiding the fatal fallacy to some
degree, but the data analysis is under psychometric perspective.

The \emph{Attractiveness} scale is part of the User Experience
Questionnaire (UEQ) inventory. Is has been vaidated by ((Bettina
Laugwitz, Theo Held, and Martin Schrepp. 2008. Construction and
Evaluation of a User Experience Questionnaire. . 63--76.
\url{https://doi.org/10.1007/978-3-540-89350-9_6})) The UEQ has
undergone basic psychometric evaluation in six studies with a single
design each.

The two scales \emph{Hedonic Quality - Identity (HQI)} and \emph{Hedonic
Quality - Stimulation (HQS)} are from the AttrakDiff2 inventory.
AttrakDiff2 underwent basic evaluation using only three Designs under
psychometric perspective (level 1 fallacy) ((Hassenzahl, M., Burmester,
M., Koller, F., AttrakDiff: Ein Fragebogen zur Messung wahrgenommener
hedonischer und pragmatischer Qualität)).

The Credibility scale \ldots{} \#\#\#\# ((HERE))

\begin{Shaded}
\begin{Highlighting}[]
\NormalTok{D\_1 }\SpecialCharTok{|\textgreater{}} 
  \FunctionTok{group\_by}\NormalTok{(Study, Scale) }\SpecialCharTok{|\textgreater{}} 
  \FunctionTok{summarize}\NormalTok{(}\AttributeTok{n\_Items =} \FunctionTok{n\_distinct}\NormalTok{(Item),}
            \AttributeTok{n\_Part =} \FunctionTok{n\_distinct}\NormalTok{(Part),}
            \AttributeTok{n\_Design =} \FunctionTok{n\_distinct}\NormalTok{(Design),}
            \AttributeTok{n\_Obs =} \FunctionTok{n}\NormalTok{()) }\SpecialCharTok{|\textgreater{}} 
  \FunctionTok{ungroup}\NormalTok{()  }\SpecialCharTok{|\textgreater{}}\NormalTok{ knitr}\SpecialCharTok{::}\FunctionTok{kable}\NormalTok{()}
\end{Highlighting}
\end{Shaded}

\begin{verbatim}
`summarise()` has grouped output by 'Study'. You can override using the
`.groups` argument.
\end{verbatim}

\begin{longtable}[]{@{}llrrrr@{}}
\toprule\noalign{}
Study & Scale & n\_Items & n\_Part & n\_Design & n\_Obs \\
\midrule\noalign{}
\endhead
\bottomrule\noalign{}
\endlastfoot
AH & nEeriness & 8 & 45 & 20 & 10800 \\
DK & nEeriness & 8 & 35 & 80 & 2800 \\
DN & Beauty & 4 & 42 & 48 & 2688 \\
DN & Hedonism & 4 & 42 & 48 & 2688 \\
DN & Usability & 4 & 42 & 48 & 2688 \\
PS & nEeriness & 8 & 39 & 87 & 2808 \\
QB & Credib & 5 & 25 & 76 & 500 \\
QB & HQI & 7 & 25 & 76 & 700 \\
QB & HQS & 7 & 25 & 76 & 700 \\
SP & Attractiveness & 6 & 40 & 66 & 1440 \\
\end{longtable}

\subsection{Statistics}\label{statistics}

Goal of the analysis is to examine in how much the psychometric fallacy
creates real biases. For this purpose, three basic psychometric
techniques were applied to several data sets. Scale reliability was
measured using Cronbach \(\alpha\). For item consistency, the corrected
item-total correlation was used and for the number of factors parallel
analysis was applied, which compares the eigenvalues to a baseline
established by bootstrapping ((REF)). For all three statistics,
functions from R package Psych were used ((REF)).

\section{Results}\label{results}

\subsection{Scale reliability}\label{scale-reliability}

\begin{Shaded}
\begin{Highlighting}[]
\NormalTok{Scale\_rel }\OtherTok{\textless{}{-}}
\NormalTok{  D\_1 }\SpecialCharTok{|\textgreater{}} 
  \FunctionTok{group\_by}\NormalTok{(Scale) }\SpecialCharTok{|\textgreater{}} 
  \FunctionTok{group\_split}\NormalTok{() }\SpecialCharTok{|\textgreater{}} 
  \FunctionTok{map\_df}\NormalTok{(alpha\_ci)}
\end{Highlighting}
\end{Shaded}

\begin{verbatim}
Some items ( Att4 Att6 ) were negatively correlated with the first principal component and 
probably should be reversed.  
To do this, run the function again with the 'check.keys=TRUE' option
\end{verbatim}

\begin{Shaded}
\begin{Highlighting}[]
\NormalTok{Scale\_rel }\SpecialCharTok{|\textgreater{}}\NormalTok{ knitr}\SpecialCharTok{::}\FunctionTok{kable}\NormalTok{()}
\end{Highlighting}
\end{Shaded}

\begin{longtable}[]{@{}llrrr@{}}
\toprule\noalign{}
Scale & Perspective & center & lower & upper \\
\midrule\noalign{}
\endhead
\bottomrule\noalign{}
\endlastfoot
Attractiveness & designometric & 0.6385208 & 0.4544574 & 0.7789188 \\
Attractiveness & psychometric & 0.3585293 & -0.1422689 & 0.6209516 \\
Beauty & designometric & 0.9680325 & 0.9554687 & 0.9777357 \\
Beauty & psychometric & 0.5818252 & 0.3527609 & 0.7300696 \\
Credib & designometric & 0.5542727 & 0.3416156 & 0.7216517 \\
Credib & psychometric & 0.4242509 & -0.0860104 & 0.7023335 \\
HQI & designometric & 0.6878708 & 0.4775373 & 0.7934202 \\
HQI & psychometric & 0.6477610 & 0.0403564 & 0.8727338 \\
HQS & designometric & 0.7377272 & 0.6009868 & 0.8189580 \\
HQS & psychometric & 0.6746067 & 0.2204059 & 0.8448004 \\
Hedonism & designometric & 0.9680615 & 0.9500491 & 0.9783806 \\
Hedonism & psychometric & 0.6258834 & 0.4276357 & 0.7842923 \\
Usability & designometric & 0.8720665 & 0.7908254 & 0.9183869 \\
Usability & psychometric & 0.6466734 & 0.4343040 & 0.7931202 \\
nEeriness & designometric & 0.8772948 & 0.8348933 & 0.9108219 \\
nEeriness & psychometric & 0.8180231 & 0.7418724 & 0.8804442 \\
\end{longtable}

\begin{Shaded}
\begin{Highlighting}[]
\NormalTok{Scale\_rel }\SpecialCharTok{|\textgreater{}} 
  \FunctionTok{ggplot}\NormalTok{(}\FunctionTok{aes}\NormalTok{(}\AttributeTok{color =}\NormalTok{ Scale,}
             \AttributeTok{label =}\NormalTok{ Scale,}
             \AttributeTok{x =}\NormalTok{ Perspective,}
             \AttributeTok{y =}\NormalTok{ center,}
             \AttributeTok{ymin =}\NormalTok{ lower,}
             \AttributeTok{ymax =}\NormalTok{ upper)) }\SpecialCharTok{+}
  \FunctionTok{geom\_point}\NormalTok{() }\SpecialCharTok{+}
  \FunctionTok{geom\_line}\NormalTok{(}\FunctionTok{aes}\NormalTok{(}\AttributeTok{group =}\NormalTok{ Scale)) }\SpecialCharTok{+}
  \FunctionTok{ylab}\NormalTok{(}\StringTok{"std. Cronbach alpha"}\NormalTok{) }\SpecialCharTok{+}
  \FunctionTok{geom\_label}\NormalTok{() }\SpecialCharTok{+}
  \FunctionTok{ylim}\NormalTok{(}\DecValTok{0}\NormalTok{,}\DecValTok{1}\NormalTok{)}
\end{Highlighting}
\end{Shaded}

\begin{figure}[H]

{\centering \pandocbounded{\includegraphics[keepaspectratio]{DMX_1_files/figure-pdf/scale-reliability-1.pdf}}

}

\caption{Cronbach alpha item-level reliability estimates compared by
perspective and scale}

\end{figure}%

\textbf{?@fig-scale-consistency} shows the Cronbach \(\alpha\) scale
reliability estimates produced by designometric and psychometric
response matrices. Overall scale reliabilities cover a broad range from
excellent to unusable. All scale reliabilities improve under the
designometric perspective, albeit, the difference ranges from barely
noticable (HQS, HQI) to very strong (Hedonism, Usability, Beauty and
Attractiveness). The most dramatic difference can be seen in Hedonism
and Beauty, which both have excellent dmx reliability, which drops to an
almost unusable level under pmx.

\subsection{Item consistency}\label{item-consistency}

\begin{Shaded}
\begin{Highlighting}[]
\NormalTok{Item\_rel }\OtherTok{\textless{}{-}}
\NormalTok{  D\_1 }\SpecialCharTok{|\textgreater{}} 
  \FunctionTok{group\_by}\NormalTok{(Scale) }\SpecialCharTok{|\textgreater{}} 
  \FunctionTok{group\_split}\NormalTok{(Scale) }\SpecialCharTok{|\textgreater{}} 
  \FunctionTok{map\_df}\NormalTok{(item\_rel)}
\end{Highlighting}
\end{Shaded}

\begin{verbatim}
Some items ( Att4 Att6 ) were negatively correlated with the first principal component and 
probably should be reversed.  
To do this, run the function again with the 'check.keys=TRUE' option
\end{verbatim}

\begin{verbatim}
\end{verbatim}

\begin{Shaded}
\begin{Highlighting}[]
\NormalTok{Item\_rel }\SpecialCharTok{|\textgreater{}} 
  \FunctionTok{ggplot}\NormalTok{(}\FunctionTok{aes}\NormalTok{(}\AttributeTok{x =}\NormalTok{ Perspective,}
             \AttributeTok{y =}\NormalTok{ r.cor)) }\SpecialCharTok{+}
  \FunctionTok{geom\_line}\NormalTok{(}\FunctionTok{aes}\NormalTok{(}\AttributeTok{group =}\NormalTok{ Item)) }\SpecialCharTok{+}
  \FunctionTok{ylab}\NormalTok{(}\StringTok{"Item{-}total correlation"}\NormalTok{) }\SpecialCharTok{+}
  \FunctionTok{geom\_label}\NormalTok{(}\FunctionTok{aes}\NormalTok{( }\AttributeTok{label =}\NormalTok{ Item)) }\SpecialCharTok{+}
  \FunctionTok{facet\_wrap}\NormalTok{(}\SpecialCharTok{\textasciitilde{}}\NormalTok{Scale, }\AttributeTok{ncol =} \DecValTok{2}\NormalTok{) }\SpecialCharTok{+}
  \FunctionTok{geom\_point}\NormalTok{(}\AttributeTok{data =} \FunctionTok{rename}\NormalTok{(Scale\_rel, }\AttributeTok{alpha =}\NormalTok{ center),}
             \FunctionTok{aes}\NormalTok{(}\AttributeTok{x =}\NormalTok{ Perspective, }
                 \AttributeTok{y =}\NormalTok{ alpha,}
                 \AttributeTok{col =} \StringTok{"Scale reliability"}\NormalTok{)) }\SpecialCharTok{+}
  \FunctionTok{geom\_line}\NormalTok{(}\AttributeTok{data =} \FunctionTok{rename}\NormalTok{(Scale\_rel, }\AttributeTok{alpha =}\NormalTok{ center),}
             \FunctionTok{aes}\NormalTok{(}\AttributeTok{x =}\NormalTok{ Perspective, }
                 \AttributeTok{y =}\NormalTok{ alpha,}
                 \AttributeTok{group =}\NormalTok{ Scale,}
                 \AttributeTok{col =} \StringTok{"Scale reliability"}\NormalTok{))}
\end{Highlighting}
\end{Shaded}

\begin{figure}[H]

{\centering \pandocbounded{\includegraphics[keepaspectratio]{DMX_1_files/figure-pdf/item-reliability-1.pdf}}

}

\caption{Cronbach alpha item-level reliability estimates compared by
perspective and scale}

\end{figure}%

\textbf{?@fig-item-reliability} shows the corrected item-total
correlations as a measure for item consistency. Beauty and Hedonism
stand out, because all items take a similar sharp drop in reliability
under pmx. To some extent this also seems to hold for Usability and
Eeriness. For Credibility, HQ-I, HQ-S and Attractiveness some items drop
under pmx, whereas others improve, with two extreme cases: Items Att4
andf Att6 are already on a very low level of reliability under dmx, but
under pmx, they even become negatively correlated. Items HQI5 and HQI6
perform poorly under dmx, but are among the overall best performing
items under pmx.

\subsection{Number of factors}\label{number-of-factors}

Often, different scales are used in combination to create a more
complete picture. It is usually the aim that a scale measures exactly
one construct (or latent variable) and that different scales measure
different constructs.

In contrast, the AttrakDiff2 questionnaire comprises two scales to
capture supposedly different aspects.

Given a response matrix, the number of factors can be estimated using
parallel analysis. Ideally, this procedure returns exactly as many
factors as there are separate scales. Here, we use parallel analysis to
assess whether the two perspectives produce the expected number of
factors, or at least agree on a number.

((MacDorman)) found that the Eeriness scale decomposes into two slightly
different aspects, summarized as ``eery'' and ``spine-tingling''.

\begin{Shaded}
\begin{Highlighting}[]
\FunctionTok{parallel\_analysis}\NormalTok{(D\_Eer, }\DecValTok{2}\NormalTok{, }\StringTok{"D"}\NormalTok{, }\StringTok{"Eeriness"}\NormalTok{)}
\end{Highlighting}
\end{Shaded}

\begin{verbatim}
`summarise()` has grouped output by 'Design'. You can override using the
`.groups` argument.
\end{verbatim}

\begin{figure}[H]

{\centering \pandocbounded{\includegraphics[keepaspectratio]{DMX_1_files/figure-pdf/nfactors-eeriness-1.pdf}}

}

\caption{Suggested number of factors for the Eeriness scale compared by
perspective}

\end{figure}%

\begin{verbatim}
Parallel analysis suggests that the number of factors =  1  and the number of components =  NA 
\end{verbatim}

\begin{Shaded}
\begin{Highlighting}[]
\FunctionTok{parallel\_analysis}\NormalTok{(D\_Eer, }\DecValTok{2}\NormalTok{, }\StringTok{"P"}\NormalTok{, }\StringTok{"Eeriness"}\NormalTok{)}
\end{Highlighting}
\end{Shaded}

\begin{verbatim}
`summarise()` has grouped output by 'Part'. You can override using the
`.groups` argument.
\end{verbatim}

\begin{figure}[H]

{\centering \pandocbounded{\includegraphics[keepaspectratio]{DMX_1_files/figure-pdf/nfactors-eeriness-2.pdf}}

}

\caption{Suggested number of factors for the Eeriness scale compared by
perspective}

\end{figure}%

\begin{verbatim}
Parallel analysis suggests that the number of factors =  1  and the number of components =  NA 
\end{verbatim}

The results suggest that under dmx only one latent variables exists,
whereas pmx produces two.

On theoretical grounds, the AttrakDiff2 inventory splits hedonistic
quality into two components, Identity and Stimulation, while the
credibility scale is a completely separate construct. We would expect
three factors to emerge.

\begin{Shaded}
\begin{Highlighting}[]
\FunctionTok{parallel\_analysis}\NormalTok{(D\_Att, }\DecValTok{3}\NormalTok{, }\StringTok{"D"}\NormalTok{, }\StringTok{"AttrakDiff and Credibility"}\NormalTok{)}
\end{Highlighting}
\end{Shaded}

\begin{verbatim}
`summarise()` has grouped output by 'Design'. You can override using the
`.groups` argument.
\end{verbatim}

\begin{verbatim}
Warning in fa.stats(r = r, f = f, phi = phi, n.obs = n.obs, np.obs = np.obs, :
The estimated weights for the factor scores are probably incorrect.  Try a
different factor score estimation method.
\end{verbatim}

\begin{figure}[H]

{\centering \pandocbounded{\includegraphics[keepaspectratio]{DMX_1_files/figure-pdf/nfactors-attrakdiff-1.pdf}}

}

\caption{Suggested number of factors for AttrakDiff inventory plus
Credibility compared by perspective}

\end{figure}%

\begin{verbatim}
Parallel analysis suggests that the number of factors =  5  and the number of components =  NA 
\end{verbatim}

\begin{Shaded}
\begin{Highlighting}[]
\FunctionTok{parallel\_analysis}\NormalTok{(D\_Att, }\DecValTok{3}\NormalTok{, }\StringTok{"P"}\NormalTok{, }\StringTok{"AttrakDiff and Credibility"}\NormalTok{)}
\end{Highlighting}
\end{Shaded}

\begin{verbatim}
`summarise()` has grouped output by 'Part'. You can override using the
`.groups` argument.
\end{verbatim}

\begin{figure}[H]

{\centering \pandocbounded{\includegraphics[keepaspectratio]{DMX_1_files/figure-pdf/nfactors-attrakdiff-2.pdf}}

}

\caption{Suggested number of factors for AttrakDiff inventory plus
Credibility compared by perspective}

\end{figure}%

\begin{verbatim}
Parallel analysis suggests that the number of factors =  1  and the number of components =  NA 
\end{verbatim}

Under a desginometric perspective, the three scales have five underlying
factors, but merging into one under pmx.

Finally, in study DN three independent scales, Hedonism, Usability and
Beauty, were used. But, parallel analysis suggests that these capture
the same latent variable under both perspectives.

\begin{Shaded}
\begin{Highlighting}[]
\FunctionTok{parallel\_analysis}\NormalTok{(D\_HUB, }\DecValTok{3}\NormalTok{, }\StringTok{"D"}\NormalTok{, }\StringTok{"Hedonism, Usability and Beauty"}\NormalTok{)}
\end{Highlighting}
\end{Shaded}

\begin{verbatim}
`summarise()` has grouped output by 'Design'. You can override using the
`.groups` argument.
\end{verbatim}

\begin{figure}[H]

{\centering \pandocbounded{\includegraphics[keepaspectratio]{DMX_1_files/figure-pdf/nfactors-HUB-1.pdf}}

}

\caption{Suggested number of factors for Hedonism, Usability and Beauty
scales compared by perspective}

\end{figure}%

\begin{verbatim}
Parallel analysis suggests that the number of factors =  1  and the number of components =  NA 
\end{verbatim}

\begin{Shaded}
\begin{Highlighting}[]
\FunctionTok{parallel\_analysis}\NormalTok{(D\_HUB, }\DecValTok{3}\NormalTok{, }\StringTok{"P"}\NormalTok{, }\StringTok{"Hedonism, Usability and Beauty"}\NormalTok{)}
\end{Highlighting}
\end{Shaded}

\begin{verbatim}
`summarise()` has grouped output by 'Part'. You can override using the
`.groups` argument.
\end{verbatim}

\begin{figure}[H]

{\centering \pandocbounded{\includegraphics[keepaspectratio]{DMX_1_files/figure-pdf/nfactors-HUB-2.pdf}}

}

\caption{Suggested number of factors for Hedonism, Usability and Beauty
scales compared by perspective}

\end{figure}%

\begin{verbatim}
Parallel analysis suggests that the number of factors =  1  and the number of components =  NA 
\end{verbatim}

\section{Discussion}\label{discussion}

Rating scales in Human Factors research are commonly used to
discriminate between poor and good design options, rank designs, choose
designs, or perform UX regression tests in continuous design cycles. Our
logical argument is that the capability of a scale to rank designs can
only be seen on multiple designs and using design-by-item response
matrices. We called it the psychometric fallacy to use person-by-item
response matrices in place. A simulation showed, that the worst case can
happen under the psychometric fallacy: excellent reliability is
reported, when it actually is very poor.

We used data from five experiments to assess the severity of the
psychometric fallacy in real practice. Overall, the results show that
the psychometric fallacy produces dramatic biases for all tested
methods. For practicioners using these scales the good news is that all
scales performed better under the correct desgnometric perspective, and
most of them even fairly well.

In contrast, item consistency and factor analysis showed that the
psychometric fallacy can lead to strong biases. Items can suddenly
become negatively correlated, as in the case of Attractiveness. The two
Hedonism scales from AttrakDiff and the Credibility scale showed an
extreme pattern, where the majority of items remained relatively stable,
whereas two items switched from poor reliability to excellent
reliability under pmx. With these patterns in mind, it is almost not
surprising that factor analysis can also produce quite different
results. Most strikingly, under dmx not a single result matched the
theoretical expectations.

\subsection{Dev time implications}\label{dev-time-implications}

In design research the target of all research is quickly changing and
expanding target. A certain swiftness and pragmatism is required to keep
up with the pace. Development of new scales is a common task, and often
it is carried out by researchers with a basic understanding of
psychometric principles, such as (item) reliability and exploratory
factor analysis.

Basic psychometric tools produce vastly different results under the
psychometric fallacy. While our study used mature scales, which had
already undergone item selection and perhaps factor analysis, we can
interpolate the consequences for future scale development.

The most severe consequence is that a scale may be developed that is not
capable of ranking designs. According to an often cited rule-of-thumb,
scale reliability should be at least .7. Three scales in our study,
Attractiveness, Credibility and HQS did not meet this criterion even
under the designometric perspective.

Interestingly, on item level Credibility and HQ-I show the same pattern,
where two items perform well under pmx, but extremely poor under dmx.
This may be a co-incidence, but a likely outcome of developing a dmx
scale under pmx is to \emph{false favor} items that are well-behaved in
ranking persons, but are inefficient for designs. To make the case,
(\textbf{Tab-rel-after-removal?}) compares dmx scale reliability on
these three scales with and without their two ill-behaved items.

\begin{Shaded}
\begin{Highlighting}[]
\FunctionTok{print}\NormalTok{(}\StringTok{"TODO"}\NormalTok{)}
\end{Highlighting}
\end{Shaded}

\begin{verbatim}
[1] "TODO"
\end{verbatim}

A pmx perspective may also \emph{false reject} items that are actually
well-behaved in ranking designs. Creating an item pool is by itself a
time-consuming process, and the psychometric fallacy can make it even
more difficult by unnecessarily rejecting items. A possible example is
the development of the BUS-11 scale, where face validity demands (and
factor analysis has confirmed) ((REF Simone)) that \emph{Privacy} is a
separate construct. Unfortunately, only one item was left after item
selection.

\subsection{Run time implications}\label{run-time-implications}

For practitioners, the good news are that if they were (or will be)
under the run-time psychometric fallacy by routinely reporting scale
reliability, they were (or will be) always better than they said. And
when they continue to use these scales in the future, the improved
precision will allow them to reduce sample sizes.

But, practitioners may not have the most efficient rating scales, yet.
Even if a false favored item is not directly harming reliability, it can
make the scale inefficient. In practice, UX scales are often deployed
during use, for example in usability tests. With a shorter scale
measures can be taken in quicker succession, for example once per task,
or everyday in a longitudinal study. It is therefore not uncommon for
practitioners to create a reduced scale, for example, when many latent
variables are involved. For some scales (Hedonism, Beauty) it is safe to
just pick three items at random, Other scales are quite mixed bags, with
the highest ranked item under pmx being the lowest ranked under dmx. If
, as is good practice, the creators of scales published pmx item-level
reliability, it becomes almost inevitable that a researcher seeking for
the three best items, will pick two of the worst.

\subsection{Criticism of individual
scales}\label{criticism-of-individual-scales}

One can rightfully argue that we used the rating scales outside their
specification, as the encounters were brief and without interaction.
This depends on the scale and on the purpose. Face processing is one of
the fastest complex mechanisms in the human mind, which makes it legit
to test it in quick succession on a screen. Similarly, the beauty
judgement of a website is known to stabilize within 500ms, and the same
can be expected for attractiveness. For more longitudinal feelings about
designs, like hedonism, usability and credibility, these encounters were
not valid. We will therefore not criticize these instruments, the point
still standing that the two perspectives produce different results.

\begin{itemize}
\tightlist
\item
  Eeriness
\item
  Beauty
\item
  Attractiveness
\end{itemize}

TODO

\subsection{Future Applications}\label{future-applications}

A founding idea in usability engineering is that a good designers has
learned to bridge the gap between the system model and the users mental
model, cognitive skills and feelings. Emerging domains are often
characterized by an innovation phase, where multiple design paths are
explored, before this diversity converges into an oligarchy of mature
sibling frameworks with established best practices and quality
principles.\\
Two emerging domains of human-technology interaction are currently
gaining significant momentum: interactive deep learning models, such as
chatbots and social robots. Both fields have in common that they tap
much stronger into the social mind of users, which, next to being a cave
of snakes, is mostly uncharted terrain. We can expect a wild growth of
theoretical concepts and instruments to measure the respective latent
variables.

((Simone))

\subsection{Towards Deep
Designometrics}\label{towards-deep-designometrics}

By comparing the two perspectives, we illustrated that designometric
analysis can fully be done with standard psychometric tools, as long as
one uses the correct response matrix. However, by reducing the
designometric box to a flat matrix, we loose all information on users.
Formally, it would even be possible to evaluate a designometric model on
the responses of a \emph{single user}, while the situation

If the cube is collapsed to a psychometric matrix, which can be used to
estimate \emph{user sensitivity}. Legitimate cases exist to use a
designometric scale for psychometric purposes. For example, an
instrument to measure trustworthiness of designs could be used to
estimate faithfulness levels of participants in a study (or a training)
on cyber security.

By flattening the designometric box one way, then the other, we still
loose information that is needed to secure that items are truly
well-behaved. In educational psychometrics \emph{differential item
functioning} is the idea that items must be fair and function the same
for every tested person. This also is a desirable property for a
designometric scale, but a statistical model for verification would need
individual parameters for participants, designs and items,
simultaneously. Schmettow(2021) proposed multi-level models for
capturing designometric situations in their full dimension, which could
be well-suited for run-time use or basic scale development.

Another consideration is that the designometric encounter may not be end
of story. For example, for comparing multi-purpose designs a researcher
may want to add tasks as fourth population of interest. With the
mentioned limitations, multi-level models extend to such a case
(Schmettow, 2016, Egan's assumption). For development-time purposes,
Generalizability Theory may provide \ldots{}

An unsolved issue is to identify exploratory methods that can operate on
deep designometric data. EFA is often used with CFA to find and confirm
candidate structures.

\bookmarksetup{startatroot}

\chapter*{References}\label{references}
\addcontentsline{toc}{chapter}{References}

\markboth{References}{References}

\phantomsection\label{refs}
\begin{CSLReferences}{0}{1}
\end{CSLReferences}




\end{document}
